\subsection{Background}

The field of traffic safety has seen huge developments since the introduction of the automobile. Initially, passive safety systems were introduced to prevent or reduce injuries in case of a crash \cite{bishop2005intelligent}. These systems - such as crumble zones, seat belts and airbags - are built in the vehicle structure and are designed to restrain the occupant and absorb energy in case of a crash. As technology and the increased use of electronics in vehicles progressed, engineers have been able to develop active safety systems that are designed to prevent a crash from happening, for example, Electronic Stability Program (ESP), Lane Departure Warning (LDW) and Automatic Emergency Braking (AEB). Active safety systems could either issue warnings to the driver, such as LDW, or make corrections to help the driver avoid a crash, such as ESP \cite{bishop2005intelligent}.  

Driver models serve as an important input in developing active safety systems: driver models can for example be used for the assessment of predictive safety benefits and for the definition of warning strategies. Driver models are used to predict human behaviour in different critical scenarios, and act upon these predictions to mitigate or reduce the severity of crashes \cite{bargman2017counterfactual}. There are different driver models with different complexities. 

One well known model is the 
car-following model presented by Gazis, Herman, and
Rothery (1961). In this model, the diver aims to keep the same speed as the  lead vehicle and adapts to speed changes faster at higher speed, low headway distances . Parameters for this model were tuned through conducting follow-the-leader experiments \cite{bexelius1968extended}. This model was later extended to account for a number of error-inducing driver behaviours and stochasticity and have been used for evaluating active safety systems, such as FCW \cite{markkula2012review}. Another way to gather data about driver behaviour for these models is through NDS.
% These models will also become important for the development of autonomous vehicles, that will most likely drive on roads among normal vehicles driven by humans. It is then crucial that an autonomous system can predict what surrounding vehicles are going to do and act accordingly. These developments should insure safer roads and better traffic safety. 

NDS is a methodology in which the participants are asked to drive as they would do in their daily life \cite{Udrive} and data are collected from vehicles equipped  with sensors and cameras. The sensors provide variables, such as speed and GPS location, while the two cameras, one pointed at the driver and the other one at the road ahead, provide videos showing the behaviour of the driver and the other road users. 

The data used in this project was collected through the second Strategic Highway Research Program (SHRP2) though a naturalistic driving study involving more than 3000 male and female volunteers, with the aim to learn how individual driver behaviour interacts with vehicle and roadway characteristics \cite{hallmark2011evaluation}. 

