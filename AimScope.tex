\subsection{Aim and Scope}
% The aim is a brief description of the assignment and its intended outcome.

As previously mentioned, the annotation of NDS videos is mostly done manually. Through developing tools to aid this process, a larger set of videos could be annotated in a shorter time, providing more variables to be used for the analysis and modelling of driver behaviour.

Throughout the course of the project, the team aims to develop a semi-automatic tool that aids the annotation of NDS, in lane-change manoeuvres involving two vehicles. The tool will be used to determine variables, such as the distance between the SV and the POV, the POV speed, the distance between the POV and the lane markings, and the positioning of the SV in the lane. The tool should be semi-automatic, meaning that the user will manually annotate features such as the lane markings and the POV width. For this reason, the tool includes a Graphical User Interface (GUI), which could support the performance of manual annotations. 

By using this tool, the accuracy of the estimated variables (e.g. speed of POV and distance between SV and POV) will increase and the time required for the annotations will decrease. As a consequence, the accuracy of the analyses to design driver models developed using NDS will improve.